\chapter{Prova}

\section{Pippo}
A seguito della crescente diffusione di apparecchi portabili quali laptop, palmari, cellulari, ecc
utilizzati in un numero sempre maggiore di luoghi pubblici e privati quali ospedali, aereoporti, uffici,
aziende e negozi si sono dovute trovare soluzioni che conciliassero necessit� diverse come interattivit�,
molteplicit� di funzioni, semplicit� di utilizzo, maneggevolezza e riduzione degli ingombri.

I maggiori problemi riguardano il compromesso tra molteplicit� di funzioni e riduzione degli ingombri,
in quanto accessori aggiuntivi e funzioni complicate implicano
risorse computazionali elevate a discapito di un maggiore consumo di energia.
In particolar modo l'utilizzo di questi dispositivi ha permesso una comunicabilit� elevatissima:
le informazioni desiderate possono essere recuperate sempre ed ovunque in pochi minuti,
grazie alla loro condivisione in reti quali internet.

Proprio per quanto riguarda lo scambio di informazioni col mondo esterno � necessario munire questi
dispositivi di interfacce per il collegamento con l'esterno.
L'utilizzo di queste interfacce influisce in modo negativo sull'autonomia del dispositivo
consumando molta energia, ci� implica che bisogna porre dei limiti sul numero e sull'utilizzo delle interfacce
per non costringere l'installazione di batterie troppo ingombranti e pi� pesanti.

Un esempio molto pratico e concreto si pu� fare su un dispositivo entrato oramai nel modo di vita di tutti noi,
il cellulare. Lo sviluppo di questo dispositivo nato inizialmente solo per effettuare telefonate � vertiginoso,
ora permette di inviare messaggi di testo, di effettuare fotografie, di personalizzare le suonerie, di scambiare
e-mail, di navigare in internet, di giocare e tant'altro; tutto ci� per� consuma energia:
 fotocamera nel caso delle fotografie o display grandi e spaziosi per i giochi;
ci� non tanto a discapito dell' ingombro grazie alla miniaturizzazione della componentistica elettronica
quanto all'eccessivo consumo energetico colpevole
di un'autonomia delle batterie minore.
Proprio per questi ultimi due elementi � invece richiesto l'opposto,
 cio� un ingombro sempre minore e un'autonomia maggiore;
� importante trovare un compromesso, in quanto per aumentare l'autonomia sarebbe sufficiente
installare batterie di dimensioni superiori che per� implica un aumento di peso del dispositivo non trascurabile.

Essendo dispositivi destinati ad un utilizzo mobile non sono accettabili i tradizionali
collegamenti tramite cavo, le nuove tecnologie permettono comunicazioni senza cavo come
 infrarosso o onde radio; nello specifico Irda per la prima tipologia, Bluetooth e Wireless LAN per la seconda.
A fronte di queste nuove tecnologie sono emerse nuove problematiche
 quali limitata copertura del servizio, tasso d'errore nella comunicazione maggiore che comporta la
ritrasmissione delle infomazioni perse, velocit� di connessione minore e consumi di energia maggiori.\\
Proprio in questo quadro si colloca tale lavoro, il quale consiste in un nuovo modo di utilizzare 
una tecnologia esistente e consolidata quale la tecnologia wireless LAN,
 nella sua versione pi� diffusa � conosciuta, lo standard 802.11b;
tale nuovo modo consente di minimizzare il consumo di energia per aumentare la durata delle batteria
 di cui ogni dispositivo portatile � fornito.

Generalmente ogni stazione wireless comunica con un server che pu� essere collegato ad internet,
o disporre egli stesso delle informazioni richieste dal dispositivo portatile.
Il nostro intento � la riorganizzazione ({\em reshaping}) del traffico,
 costituito dalle informazioni dirette verso il dispositivo portatile direttamente sul server,
questo ci permette un risparmio energetico delle stazioni wireless senza modificare
 l'architettura attualmente esistente.

Sostenere una comunicazione costa molto in termini di energia consumata,
per tale ragione � necessario evitare comunicazioni inutili come l'ascolto di traffico destinato ad altri
o l'attesa continua di informazioni ({\em pacchetti}) che per qualche ragione non vi sono attualmente.
A seguito degli elementi sopra elencati ci troviamo di fronte ad un problema di {\em Dinamic Power Managemant},
cio� di variare dinamicamente il consumo di energia della stazione wireless a seconda che debba o meno
scambiare informazioni.
Principalmente i metodi di power management cercano di mantenere la scheda wireless il pi� possibile
 in uno stato di basso consumo, perci� � importante il modo in cui le viene passato il traffico.
Se un' applicazione invia pachetti sporadicamente, tiene attiva per molto tempo la scheda wireless inutilmente;
sar� allora necessario riorganizzare il traffico ad un livello sottostante, raggruppandolo ed inviandolo solo
quando � veramente corposo, da rendere conveniente il risveglio della scheda.

A fronte delle considerazioni soprastanti, abbiamo deciso di spostare il problema dalle stazioni wireless al server
principalmente per due motivi:
innanzi tutto il server � collegato all'alimentazione di rete per cui non ha i problemi di autonomia delle batterie,
e poi perch� ha la visione del traffico globale diretto verso tutte le stazioni wireless a lui collegate.
Proprio tramite la visione globale del traffico � possibile un reshaping in modo che ogni stazione riceva solamente
le informazioni a lei destinate o dirette a tutti ({\em broadcast}), cosa che allo stato attuale non �;
infatti ogni stazione � sensibile anche alle informazioni dirette ad altri provocando un inutile consumo di energia.
Il reshaping consiste nell'invio di blocchi di informazioni dirette verso una sola stazione, mettendola in stand-by
quando non ve ne sono; attualmente esiste una sorta di modo di lavoro in stand-by analizzata nel dettaglio in seguito,
ma al risveglio una stazione pu� riceve comunque le informazioni altrui; il nostro intento � disaccopiare
il pi� possibile la visione delle informazioni altrui sino a farle scomparire.

%alcuni risultati sul risparmio
\section{CiuCiu}
Prima di analizzare nel dettaglio come viene effettuato il reshaping � necessario vedere le caratteristiche
principali del protocollo 802.11; proprio a tal fine il capitolo 2 � stato concepito per introdurre
le problematiche relative allo scambio di informazioni in una rete senza fili e alla loro risoluzione.
Inoltre sempre nel capitolo 2 viene mostrato lo stato attuale del protocollo da noi utilizzato,
le implementazioni riguardanti il power management e la monitorizzazione del consumo di potenza.

Nel capitolo 3 verr� invece introdotto nel dettaglio il principio di funzionamento e l'idea che sta alla
base dello scheduler da noi implementato per la riduzione del consumo energetico; inoltre verr�
spiegata la relativa implementazione analizzando le singole sezioni del codice sorgente.
 
Nel capitolo 4 � riportato il modo in cui sono stati condotti i test, la realizzazione di una semplice
architetture sperimentale per la monitorizzazione dei dati relativi al consumo di potenza e
l'analisi dei dati da essa prodotti.

Infine nel capitolo 5 vi sono alcune considerazioni sul lavoro realizzato, i benefici introdotti
e le ulteriori migliorie introducibili sempre in questa direzione.
